%!TEX root = ../main.tex
%******************************
%	 Discussion 
%*****************************

\chapter{Conclusion and future directions}  

\vspace{-5mm}

My work focused on the statistical quantification of transcriptional noise in biological systems such as the activation response of CD4\plus{} T cells. Firstly, in collaboration with Celia P. Martinez-Jimenez, I used scRNA-Seq data of CD4\plus{} T cells to identify an age-related increase in transcriptional noise within a set of immune response genes (see \textbf{Chapter 2}).   
While technological and computational advances of the recent years facilitate the quantification of biological noise across a range of cell types and tissues, major challenges remain regarding robust measurement, mathematical modelling and experimental validation. Here, I will discuss the my work in the light of current challenges in the field of scRNA-Seq when measuring biological variation across multiple cells.

\newpage

\section{Technologies to study the biological role of noise}

We explored the effect of ageing on transcriptional noise during immune activation. Early immune activation induces a transcriptional switch from stochastic to regulated gene expression coupled with a reduction in transcriptional variability. These dynamics and, more importantly, a set of immune-related response genes are conserved during evolution. While ageing only shows subtle effects on the overall transcriptomic profiles of individual cells, we observe a strong increase in expression variability in the core set of immune response genes during ageing. Therefore, transcriptional variability is a largely unexplored factor of organismal ageing. This finding has been validated by several studies \citep{Enge2017, Angelidis2018, Cheung2018} adding the increase in transcriptional noise to the list of ageing-associated physiological effects.\\

Our study uncovered transcriptional noise to disturb the dynamic response of an otherwise tightly regulated system. The systematic analysis of how transcriptional noise globally influences other cellular systems such as the developing embryo or disease onset is still lacking. The discovery of heterogeneous gene expression lead to the identification of an underlying cell fate commitment process in the 4-cell stage embryo \citep{Goolam2016}. Mohammed \emph{et al.}, 2017 identified global changes in transcriptional noise during early mouse embryo development that correlate with the plasticity of cell populations. Pluripotent cells tend to display noisier gene expression compared to committed cells \citep{Mohammed2017}. Nevertheless, technical limitations restricted the analysis to few hundreds of cells and specific tissues per embryo. With newly developed combinatorial indexing approaches, hundreds of thousands of transcriptomes can be generated in parallel \citep{Cao2017}. This allows an unbiased detection of all major cell-types during (e.g.) embryonic development which in turn offers a great resource to perform systematic comparisons of transcriptional noise between tissues and time-points. Major drawbacks of this approach would be the reduced sequencing depth and the inability to validate the global change in variability as discussed below.\\

In \textbf{Section \ref{sec2:droplet}}, I tested for changes in expression variability between the pre-somitic and the somitic mesoderm of the developing mouse embryo. Interestingly, this analysis revealed heterogeneous up-regulation of lineage-associated genes that are later on expressed in defined tissues. This shows that testing for changes in expression variability can lead to the identification of uncharacterised, early commitment processes during embryogenesis. Nevertheless, scRNA-Seq data does not allow to assay the underlying transcriptional regulation that induces heterogeneous expression of these genes. It is therefore impossible to say whether heterogeneity in expression is induced by molecular noise or driven by deterministic processes.\\

So far, quantification of expression noise on a genome wide scale is only possible by scRNA-Seq. This raises the question if noise that is detected on the mRNA level propagates to form fluctuations in proteins which are the final driver for phenotypic variations between individual cells. Reports have been published that show a reduction of transcriptional noise during nuclear export of mRNAs \citep{Battich2015a, BaharHalpern2015a} indicating the possibility that studying biological noise on the mRNA level is further buffered in the cytoplasm by mechanisms such as miRNA-based degradation \citep{Schmiedel2015}. In recent years, technologies have been developed to measure protein abundance in single cells in high-throughput and high-content based approaches. \Gls{CyTOF} has been introduced as a single-cell technology to measure multiple proteins within hundreds of thousands of cells. For this, antibodies against membrane bound and intracellular proteins are labelled with transition element isotopes and quantified via mass spectroscopy. So far, the main application of CyTOF has been to identify immune cell dynamics \citep{Bendall2011}. To add the spatial component to mass cytometry, Giesen \emph{et al.}, 2014 developed imaging mass cytometry to obtain spatial distributions of 32 proteins in breast cancer samples \citep{Giesen2014}. A similar approach has been introduced by Gut \emph{et al.}, 2018 where off-the shelf antibodies are used to spatially resolve protein expression. During 20 rounds of primary and fluorescently-labelled secondary antibody staining, multiplexed read-outs of protein positions can be obtained from individual cells \citep{Gut2018}. The spatial detection of proteins has been extended by simultaneously measuring mRNA transcripts by isotope tagging \cite{Schulz2018}. These approaches allow (i) quantification of protein expression noise, (ii) spatially-resolved inter- and intra-cellular variations of protein abundance and (iii) the assessment of noise propagation from the mRNA to protein level. To further enhance the connection between mRNA and protein noise, and chromatin state and mRNA noise, multi-omics technologies need to advance in precision and scalability.

\newpage

\section{Confounding effects when measuring noise}

We used the BASiCS analysis framework to quantify and compare measures of transcriptional noise in the immune response of CD4\plus{} T cells. By incorporating reads of synthetic RNA spike-in molecules, BASiCS quantifies and removed technical noise from the total transcriptional variation. Throughout this thesis, we used the over-dispersion parameter $\delta_i$ to capture biological variability in expression after removal of unwanted technical variation. Furthermore, to account for experimental designs where cells were captured in multiple replicates, BASiCS scales technical noise batch-specifically  \citep{Vallejos2015BASiCS}. We described a genes' mean expression as an additional factor that confounds testing changes in over-dispersion. Therefore, we extended the BASiCS framework to derive residual over-dispersion estimates that show no correlation to mean expression (see \textbf{Chapter 3}). \\

By applying this model to capture changes in variability over the differentiation time-course of spermatogenesis, I observed that the strength of transcriptional changes over time introduce an additional confounding factor that, so far, has not been accounted for. I will therefore discuss a variety of confounding factors that influence the quantification of transcriptional noise grouped into experimental and technical effects.

\subsection{Experimental confounding factors}

Transcriptional noise as defined in \textbf{Box 1} can only be measured in truly homogeneous populations of cells. Previous studies that quantified transcriptional variability from scRNA-Seq data either sequenced mESCs (e.g.~\citep{Kolodziejczyk2015cell}), primary chicken erythroid progenitor cells \citep{Richard2016}, a murine multipotent hematopoietic precursor cell line \citep{Mojtahedi2016} or CD4\plus T cells \citep{Martinez-jimenez2017}. With the development of technologies that capture thousands of cells in an unbiased way, structured heterogeneity presents the major source of cell-to-cell variations in expression. As shown in \textbf{Section \ref{sec2:droplet}}, one relies on clustering approaches to identify homogeneous populations of cells that can be compared when testing for changes in transcriptional variability. It is therefore also crucial to understand the underlying biology that causes structured heterogeneity to avoid including low quality cells into the analysis. For example, Ibarra-Soria \emph{et al.} identified a small intermediate population between pre-somitic and somitic mesoderm with unknown identity \citep{Ibarra-Soria2018}. It is recommended to remove such cells from analysis to avoid any unknown biological heterogeneity that confounds biological noise.

\newpage
 
As shown in \textbf{Section \ref{variability_over_PT}}, quantification of transcriptional noise is also heavily influenced by the underlying differentiation programmes of otherwise homogeneous cell populations, as exemplified by the differentiation process of spermatogenesis. After extensive quality control and clustering, the remaining variation in germ cell populations is dictated by genes that strongly and abruptly change their expression levels (e.g.~\textit{Prm1}). This observation is in line with previous reports on how the cell-cycle state of each cell masks underlying population structure \citep{Buettner2015}. For each gene $i$, the \gls{scLVM} captures (e.g.) the cell cycle associated component $\hat{y}_i$ and allows the derivation of corrected counts $y^{\ast}$ by substracting this effect from the observed count $y_i$: $y^{\ast}=y_i-\hat{y}_i$. This correction can therefore be seen as a regression approach to correct for a specific confounding effect (e.g.~cell cycle). This idea can be transferred to the BASiCS framework. In addition to correcting the mean expression effect, the model can be extended to perform a semi-parametetric regression between the over-dispersion parameter and a measure of association to differentiation . This measure in the simplest case can be parameters of a regression fit between each cells' expression level and the ordering of cells along the differentiation time-course. \\

\subsection{Technical confounding factors}

\Gls{scRNA-Seq} is prone to high technical noise due to the low starting amounts of RNA transcripts that are first captures, reverse transcribed, pre-amplified, prepared for sequencing and sequenced. Only around 10\%-20\% of all transcripts are captured in each individual cell leading to high levels of technical noise. Furthermore, amplification biases exponentially enhance noise introduced by variation in capture efficiency. These biases are minimized by the introduction of \glspl{UMI} that allow the direct quantification of transcript abundance \cite{Islam2014}. In preliminary analyses to study parameter robustness as displayed in \textbf{Section \ref{sec2:parameter_stabilization}}, we observed that the UMI data \citep{Zeisel2015} resulted in generally more rubust estimates compared to non-UMI data (e.g.~CD4\plus T cells, \citep{Martinez-jimenez2017}).\\

The incorporation of UMIs into droplet-based scRNA-Seq technologies facilitates a robust estimation of transcriptional variability. On the other hand, these high-throughput methods came at the price of reduced sequencing depth, the inability to quantify technical noise via RNA spike-ins and often reduced replication. 

Technical noise
As discussed above, a variety of measures exist to quantify variability in mRNA and protein abundance for populations of single cells. Due to low starting amounts in sequencing based technologies, technical noise is a major contributor to overall variation in the data. This effect is pronounced for lowly expressed genes making it harder to estimate correct noise parameters for these genes. Without multiple replicates per condition, robust assessment and testing of changes in variability is impossible.\\

Sequencing depth\\
Replication\\
Doublets

To summarise, it is therefore crucial to ...

\section{Measures to quantify noise}

 The distribution of pairwise distances between cells has therefore been used to describe this property(Mohammed et al. 2017). More robust measurements are nevertheless still missing.
 
 Besides empirical estimations of noise in biological systems (e.g. CV2), the development of theoretical models to describe intrinsic and extrinsic noise advanced in recent years(Fu Pachter 2016). Nevertheless, these models have not been extended to learn transcription parameters (burst size and burst frequency) or to incorporate cellular consequences of noise (e.g. cell fate decisions).

\section{Experimental validation and manipulation of noise}

Classically, unicellular systems were employed to study noise. In these systems, genetic alterations allowed the modulation of transcriptional and translational variability(Raser O’Shea 2005; Ozbudak et al. 2002; Hornung et al. 2012). Specifically, changing promoter architecture shows strong alterations of expression noise(Jones et al. 2014; Sharon et al. 2014). Furthermore, different ways of general and targeted perturbation of transcriptional noise in cell populations has been discussed in Dueck et al.(Dueck et al. 2016). So far, perturbation of biological noise in higher organisms has not been experimentally described yet.\\


Specific targeting of mRNAs on a post-transcriptional level.\\

Inability to change global variability. \\


\section{Future approaches for Bayesian models in scRNA-Seq data}

One solution for this includes error propagation approaches used in Bayesian statistics(Vallejos et al. 2016). Furthermore, these single gene measures are not optimal to quantify the global noise status of a whole population of cells.

Multi-view learning, MOFA\\
Variational approaches\\
Incoporation of translational dynamics into generative models

