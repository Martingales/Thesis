%!TEX root = ../thesis.tex
% ******************************************
% 			Thesis Appendix A 
% ******************************************

\chapter{Experimental methods} 

\section{Ageing increases transcriptional noise in CD4$^+$ T cell activation}
\label{appA.1}

\subsection{Mouse material}

CAST/EiJ male mice were maintained under specific pathogen-free conditions at the University of Cambridge, CRUK – Cambridge Institute under the auspices of a UK Home Office license. Inbred wild-type C57/BL6 mice were purchased from Charles River UK Ltd (Margate, United Kingdom). Animals were euthanized in accordance with Schedule 1 of the Animals (Scientific Procedures) Act 1986. Each animal used was macroscopically examined. Animals with lesions or phenotypic alterations in their internal organs were discarded. \\

\subsection{CD4$^+$ T cell isolation}
\label{appA.1:isolation}

Unstimulated CD4$^+$ T cells were purified from dissociated mouse spleens using EASY cell strainer (30 $\mu$m, Greiner BioOne), cell separation media (lympholyte, \#{}CL5035) and the CD4$^+$ CD62L$^+$ T Cell Isolation Kit II (Miltenyi Biotec, \#{}130-093-227). Flow cytometry confirmed that 96.4\% of the isolated CD4$^+$ T cells were naive in young B6 \textbf{(Fig.~\ref{fig1:characterization}D)}. Naive CD4$^+$ T cells formed a single, high-purity population in young animals. Old animals had a small population of CD4$^+$ T cells with slightly elevated CD44 levels, reduced CD62L expression, and attenuated activation dynamics \textbf{(Fig.~\ref{fig1:characterization}E-G)}; their removal did not impact the results presented in chapter 1 \textbf{(Fig.~\ref{fig1:validation}D)}.\\

Purified unstimulated CD4$^+$ T cells were cultured in IMDM medium (GIBCO, \#{}21980-032) supplemented with 10\% Fetal Bovine Serum (Life Technology, \#{}10500064), 1 $\mu$g/mL Penicillin/Streptomicin (Life Technology, \#{}15070063), and 50 $\mu$M 2-mercaptoethanol (Gibco, \#{}31350-010). Cells were seeded into 96-well plates coated for 1h at 37$^|circ$C with anti-CD3$\epsilon$ (1 $\mu$g/ml, clone: 145-2C11, eBioscience, \#{}16-0031-82) and anti-CD28 (3 $\mu$g/ml, clone: 37.51, eBioscience, \#{}16-0281-82) at a density of 80,000-120,000 cells/ml, and then cultured in a total volume of 100 $\mu$l media that did not contain cytokines or additional antibodies.  

All cells were cultured in a humidified incubator at 37$^|circ$C, with 5\% CO2. Unstimulated and activated CD4$^+$ T cells were then immediately collected and loaded on a 5–10$\mu$m Auto Prep Integrated Fluidic Circuit (IFC; Fluidigm, San Francisco, CA) to capture single cells using the C1 Single cell Auto Prep System (Fluidigm). All the IFCs were visually inspected, and wells with multiple cells or cell debris were identified per instructions of the manufacturer (PN 101-2711 A1 White Paper). Upon cell capture, reverse transcription and cDNA amplification were performed using the SMARTer PCR cDNA Synthesis Kit (Clontech) and the Advantage 2 PCR Kit (Clontech). ERCC spike-in RNA (Ambion) (1 $\mu$L diluted at 1:50,000) was added to the C1 lysis mix. All the capture sites were included for the RNA-seq library preparation, and wells identified above as multiple cells or containing debris were removed during computational analysis.

\subsection{Flow cytometry}
\label{appA.1:FACS}

Unstimulated CD4$^+$ T cells were purified from spleens of young and old C57/BL6 mice (see above). Isolated cells were, directly or after 3h activation in vitro (see above), incubated with TruStainfcX (anti-mouse CD16/32, clone:93, BioLegend) before staining with immunofluorescence conjugated antibodies against murine CD4 (clone: RM4-5, BioLegend), CD44 (clone: IM7, BioLegend), CD62L (clone: MEL-14, BioLegend), CD25 (clone: 3C7, BioLegend), CD69 (clone: H1.2F3, BioLegend), CD127 (clone: A7R34, BioLegend), and KLRG1 (clone: 2F1, BD Biosciences). Cell viability was determined using Fixable eFluor 780 viability dye (eBioscience). Data were acquired on a 5-laser Aria IIu SORP instrument (BD Biosciences) and data analysis was performed using FlowJo software (Tree Star).\\

Naive and effector memory CD4$^+$ T cells were purified from spleens of both young and old C57/BL6 mice by FACS.  Briefly, spleens were harvested from both young and old animals and single cell suspensions were obtained by meshing through a cell strainer (70 $\mu$m). B cells were depleted from cell suspensions by MACS using CD19 microbeads (Miltenyi Biotec, \#{}130-052-201) and red blood cells were lysed with RBC lysis buffer (Biolegend, \#{}B205551). The enriched cell fraction was then stained with Fixable eFluor 780 viability dye (eBioscience) following by Fc receptor blocking with TruStain fcXTM (clone: 39, Biolegend) and subsequent staining with a panel of fluorescence-conjugated antibodies against CD4 (clone: RM4-5, BioLegend), CD44 (clone: IM7, BioLegend), CD62L (clone: MEL-14, BioLegend), CD24 (clone: M1/69, BioLegend), Qa2 (clone: 695H1-9-9, BioLegend), CD69 (clone: H1.2F3, BioLegend) and PD-1 (clone: RMP1-30, BioLegend).  Stained cells were immediately sorted using a 5-laser Aria IIu SORP instrument (BD Biosciences) with the stringent gating strategy described in \textbf{Fig.~\ref{fig1:FACS}}. 

\subsection{RNA-Seq library preparation and sequencing}
\label{appA.1:RNA-Seq}

Single-cell RNA-Seq libraries were prepared using standard Fluidigm protocol (\# PN 100-7168 K1) based on SMARTer chemistry and Illumina Nextera XT (Illumina) using paired-end 125bp sequencing on Illumina HiSeq2500. Each RNA-seq library was sequenced to a typical depth of 1.3 million reads on average. To account for potential batch effects, for each experimental condition, two biological replicates were prepared using independent C1 IFCs.

\newpage

\section{Droplet based single-cell RNA sequencing of mouse testis}
\label{appA.2}

\subsection{Mouse material}

All animals were housed in the Biological Resources Unit (BRU) in the Cancer Research UK – Cambridge Institute under Home Office Licences PPL 70/7535 until February 2018 and PPL P9855D13B from March 2018. C57BL/6J animals were purchased from Charles River UK Ltd (Margate, United Kingdom) and the Tc1 mouse line was obtained from Dr. E. Fisher and Dr. V. Tybulewizc \citep{ODoherty2005} and maintained by breeding female Tc1 mice to male (129S8 x C57BL/6J) F1 mice. Littermates that did not inherit human chromosome 21 in these crosses (Tc0) were used as control animals.
 
\subsection{Fluorescence-activated cell sorting of spermatogenic cell populations}
\label{appA.2.sorting}

Spermatogenic cell populations were isolated from adult mouse testes as described in Ernst \emph{et al.} \citep{Ernst2016}. In brief, the albuginea was removed and tissue was incubated in dissociation buffer containing 25 mg/ml Collagenase A, 25 mg/ml Dispase II and 2.5 mg/ml DNase I for 30 minutes at 37$^\circ$C. Enzymatic digestion was quenched with Dulbecco’s Modified Eagle Medium (DMEM, Gibco) supplemented with 10\% Fetal calf serum (FCS, 10270106, Gibco). Cells were resuspended at a concentration of 1 million cells per ml and stained with Hoechst 33342 (H3570, ThermoFisher Scientific) at a final concentration of 5 µg/ml for 45 minutes at 37$^\circ$C. Cells were resuspended in PBS containing 1\% FCS and 2 mM EDTA and propidium iodide was added to a final concentration of 1 µg/ml prior to sorting. \\
Cells were sorted on an Aria IIu cell sorter (Becton Dickinson) using a 100 \textmu{}m nozzle. Hoechst was excited with a UV laser at 355nm and fluorescence was recorded with a 450/50 filter (Hoechst blue) and 635LP filter (Hoechst red). Primary spermatocytes (4N) and round spermatids (1N) were sorted and collected in PBS containing 1\% FCS and 2 mM EDTA. 

\subsection{Total RNA-Seq from bulk samples}
\label{appA.2.bulk}

Testes from prepubertal mice ranging between postnatal day 6 and 35 were flash frozen or directly used for RNA extraction using Trizol (Thermo Fisher, 15596026) following manufacturer’s instructions. Purified RNA was DNase-treated using the TURBO DNA-free Kit according to manufacturer’s instructions (Thermo Fisher, AM1907) and RNA quality was assessed using the Agilent Tapestation RNA Screentape. 800 ng of DNA-depleted RNA were used for RNA-Seq library preparation using the TruSeq Stranded Total RNA Library Kit with Ribo-Zero Gold for cytoplasmic and mitochondrial ribosomal RNA removal according to manufacturer’s instructions (Illumina, RS-122-2303). Libraries were then sequenced on Illumina HiSeq2500 using a paired-end 125bp run. 

\subsection{10X Genomics Single-cell RNA-Seq}

Mouse testes were enzymatically dissociated as described above and 34 µl of single-cell suspension at a concentration of ~297,000 cells/ml was loaded into one channel of the ChromiumTM Single Cell A Chip (10X Genomics\textsuperscript{\textregistered}), aiming for a recovery of 4000-5000 cells. The Chromium Single Cell 3’ Library \& Gel Bead Kit v2 (10X Genomics\textsuperscript{\textregistered}, 120237) was used for single-cell barcoding, cDNA synthesis and library preparation, following manufacturer’s instructions according to the Single Cell 3’ Reagent Kits User Guide Version 2, Revision D. Libraries were sequenced on Illumina HiSeq2500 using a paired-end run sequencing 26bp on read 1 and 98bp on read 2. 

\subsection{Histology}
Testes were fixed in neutral buffered formalin (NBF) for 24 hours, transferred to 70\% ethanol, machine processed and paraffin embedded. Formalin-fixed paraffin-embedded (FFPE) sections of 3\textmu{}m thickness were used for all histological stains and immunohistochemistry (IHC). For Periodic Acid Schiff (PAS) stainings slides were dewaxed, washed in water and placed in 0.5\% Periodic Acid (Sigma P0430) for 5 minutes. After three washes in ultra-pure water, slides were placed in Schiff reagent (Thermo Fisher Scientific, J/7300/PB08) for 15-30 minutes in a closed container and washed again three times in ultra-pure water. Counterstain was performed using Mayers Haematoxylin (Thermo Fisher Scientific, LAMB/170-D) for 40 seconds followed by rinsing in tap water, dehydration and mounting. IHC was performed on FFPE sections using the Bond\textsuperscript{TM} Polymer Refine Kit (DS9800, Leica Microsystems) on the automated Bond Platform. Anti-phospho-Histone H3 (Ser10) (pH3) antibody (Upstate, 06-570, 1:200 dilution) was used with DAB Enhancer (Leica Microsystems, AR9432) and heat-induced epitope retrieval was performed for 10 minutes at 100$^\circ$C on the Bond platform with sodium citrate. All slides were scanned using Aperio XT (Leica Biosystems) and PH3 intensities were quantified using the Aperio eSlide Manager (Leica Biosystems). 

\subsection{Low cell number chromatin profiling using CUT\&{}RUN (Cleavage under targets and release using nuclease)}
\label{appA.2.CnR}

\emph{In situ} chromatin profiling of FACS-purified spermatogenic cell populations was performed according to Skene \emph{et al.} \citep{Skene2018}. In brief, spermatocytes and spermatids were sorted as described above and collected in PBS. Cells were spun down at 600 g for 3 minutes in swinging-bucket rotor and washed twice with 1.5 ml Wash buffer (20 mM HEPES-KOH (pH 7.5), 150 mM NaCl, 0.5 mM Spermidine and 1X cOmplete\texttrademark{} EDTA-free protease inhibitor cocktail (04693159001, Roche)). During the cell washes, concanavalin A-coated magnetic beads (Bangs Laboratories, cat. No BP531) (10 \textmu{}l per condition) were washed twice in 1.5 mL Binding Buffer (20 mM HEPES-KOH (pH 7.5), 10 mM KCl, 1mM CaCl, 1mM MnCl2) and resuspended in 10 \textmu{}l Binding Buffer per condition. Cells were then mixed with beads and rotated for 10 minutes at room temperature (RT) and samples were split into aliquots according to number of antibodies profiled per cell type. We used 20,000-30,000 spermatocytes and 40,000-60,000 spermatids per chromatin mark.\\ 

Cells were then collected on magnetic beads and re-suspended in 50 \textmu{}l Antibody Buffer (Wash buffer with 0.05\% Digitonin and 2 mM EDTA) containing one of the following antibodies in 1:100 dilution: H3K4me3 (Millipore 05-1339 CMA304, Lot2780484) and H3K9me3 (Abcam, ab8898, Lot GR306402-1). Cells were incubated with antibodies for 10 minutes at RT and then washed once with 1 ml Digitonin buffer (Wash buffer with 0.05\% Digitonin). For the mouse anti-H3K4me3 antibody, samples were incubated with a 1:100 dilution in Digitonin buffer of secondary rabbit anti-mouse antibody (Invitrogen, A27033, Lot RG240909) for 10 minutes at RT and then washed once with 1 mL Digitonin buffer. Samples were then incubated with 700 ng/ml ProteinA-MNase fusion protein (kindly provided by Steven Henikoff) for 10 minutes at room temperature followed by two washes with 1 ml Digitonin buffer. Cells were then resuspended in 100 µl Digitonin buffer and cooled down to 4$^\circ$C before addition of CaCl$_2$ to a final concentration of 2 mM. Targeted digestion was performed for 30 minutes on ice until 100 µl of 2X STOP buffer (340 mM NaCl, 20 mM EDTA, 4 mM EGTA, 0.02\% Digitonin, 250 mg RNase A, 250 \textmu{}g Glycogen, 15 pg/ml yeast spike-in DNA (kindly provided by Steven Henikoff)) were added. Cells were then incubated at 37$^\circ$C for 10 minutes to release cleaved chromatin fragments, spun down for 5 minutes at 16,000 g at 4$^\circ$C and collected on magnet. Supernatant containing the cleaved chromatin fragments was then transferred and cleaned up using the Zymo Clean \& Concentrator Kit.
Library preparation was performed using the ThruPLEX\textsuperscript{\textregistered} DNA-Seq Library Preparation Kit (R400407, Rubicon Genomics) with a modified Library Amplification programme: Extension and cleavage for 3 minutes at 72$^\circ$C followed by 2 minutes at 85$^\circ$C, denaturation for 2 minutes at 98$^\circ$C followed by four cycles of 20 seconds at 98$^\circ$C, 20 seconds at 67$^\circ$C and 40 seconds at 72$^\circ$C for the addition of indexes. Amplification was then performed for 12-14 cycles of 20 seconds at 98$^\circ$C and 15 seconds at 72$^\circ$C. Average library size was tested on Agilent 4200 Tapestation using a DNA1000 High Sensitivity Screentape and quantification was performed using the KAPA Library Quantification Kit (Kapa Biosystems). CUT\&{}RUN libraries were sequenced on a HiSeq2500 using a paired-end 125bp run. 

