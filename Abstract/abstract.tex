% ***********************************
% **** Thesis Abstract 
% ***********************************

\begin{abstract}
Transcriptional noise is an intrinsic feature of cell populations and plays a driving role in mammalian development, tissue homoeostasis and immune function. While expression heterogeneity, a phenotypic readout of transcriptional noise, has been broadly studied in prokaryotic model systems or by profiling individual genes, whole-transcriptome studies in mammalian systems are rare. The development of single-cell RNA sequencing technologies introduced powerful tools to investigated transcriptional differences between individual cells, therefore allowing the in-depth characterization of expression variability. In this thesis, I used single-cell RNA sequencing data to understand transcriptional variability  and enhanced a statistical model avoid confounding effects when estimating such variability. First, I profiled individual transcriptomes  of CD4\plus{} T cells to identify a global decrease in transcriptional variability upon immune activation. By extending this analysis across two sub-species of mice, I identified an evolutionarily conserved set of immune response genes for which expression increases in variability during ageing. Due to a strong confounding effect between mean expression and transcriptional variability, this analysis was restricted to genes that are similarly expressed across the tested conditions. Therefore, I extended the existing BASiCS framework to avoid the aforementioned confounding factor. Within this Bayesian framework, I introduced a joint prior effecting mean expression and variability parameters to calculate the mean-expression-independent residual over-dispersion for each gene. This measure allows me to statistically assess changes in variability even for genes with differences in mean expression between conditions. Finally, I used this model to identify temporal changes in variability over the time-course of spermatogenesis. This unidirectional differentiation process involves several complex steps to form mature sperm from spermatogonial stem cells. Unbiased sampling of cells from adult testes and comparing this population to cell type compositions at juvenile stages allowed me to dissect the main regulatory programmes during spermatogensis. When profiling changes in variability across this developmental time-course, peaks in variability are caused by rapid changes in gene expression along the differentiation trajectory. This thesis provides a deeper understanding of technical and biological factors that drive transcriptional variability and offers a basis for future research to characterize its role in health and disease.
\end{abstract}
