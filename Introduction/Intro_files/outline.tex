%!TEX root = ../intro.tex
%******************************
%	 Outline 
%*****************************


\section{Outline}

The overarching topic of this thesis is the quantification and interpretation of transcriptional noise as measured by scRNA-Seq. 
\textbf{Chapter 2} presents an initial experiment to study how  transcriptional noise effects the immune system. 
We see that transcriptional noise increases across multiple immune response genes during ageing which therefore could explain a disrupted immune response in older individuals. 
This finding has been published in the following paper:

\begin{Abstract}
\hspace{-5mm} Celia P. Martinez-Jimenez$^\ast$, Nils  Eling$^\ast$, Hung-Chang Chen, Catalina A. Vallejos, Aleksandra Kolodziejczyk, Frances Connor, Lovorka Stojic, Tim F. Rayner, Michael J. T. Stubbington, Sarah A. Teichmann, Maike de la Roche, John C. Marioni, Duncan T. Odom. 
Ageing increases cell-to-cell transcriptional variability upon immune stimulation. \emph{Science}, 1436: 1433-1436, 2017, \\
($^\ast$ equal contributions) 
\end{Abstract}

Studying changes in variability between two conditions was restricted to genes that did not change in mean expression due to a strong confounding between variability and mean expression. 
In \textbf{Chapter 3}, I therefore extended the statistical framework from chapter 2 to correct for this confounding effect. 
This correction leads to (i) a stabilisation of model parameters, (ii) expansion of the gene set that can be tested for changes in variability and (iii) a novel way of interpreting transcriptional dynamics. 
This project has been published as:

\begin{Abstract}
\hspace{-5mm} Nils Eling, Arianne C. Richard, Sylvia Richardson, John C. Marioni, Catalina A. Vallejos. \\
Correcting the Mean-Variance Dependency for Differential Variability Testing Using Single-Cell RNA Sequencing Data. \emph{Cell Systems}, 7: 284-294, 2018
\end{Abstract}

The extended model offers the unique opportunity to study changes in variability across multiple cell types even when mean expression changes. 
In \textbf{Chapter 4}, I apply the newly developed model to test changes in variability over pseudo-time. 
For this, droplet-based scRNA-Seq data of mouse spermatogenesis was used to dissect the transcriptional dynamics during this developmental process. 
Parts of the study are available online as:

\begin{Abstract}
\hspace{-5mm} Christina Ernst$^\ast$, Nils Eling$^\ast$, Celia P. Martinez-Jimenez, John C. Marioni, Duncan T. Odom. 
Staged developmental mapping and X chromosome transcriptional dynamics during mouse spermatogenesis. \emph{bioRxiv}, 2018, ($^\ast$ equal contributions)
\end{Abstract}

Finally, I will discuss current challenges in modelling transcriptional noise from scRNA-Seq data and experimental strategies to modulate expression variability.

\newpage

\section{Other contributions}

Contributions to papers that are not discussed in this thesis are as follows:\\

\begin{Abstract}
\hspace{-5mm} Kaia Achim$^\ast$, Nils Eling$^\ast$, Hernando Martinez Vergara, Paola Yanina Bertucci, Jacob Musser, Pavel Vopalensky, Thibaut Brunet, Paul Collier, Vladimir Benes, John C. Marioni, Detlev Arendt. 
Whole-Body Single-Cell Sequencing Reveals Transcriptional Domains in the Annelid Larval Body. \emph{Molecular Biology and Evolution}, 35: 1047-1062, 2018, ($^\ast$ equal contributions)
\end{Abstract}

\begin{Abstract}
\hspace{-5mm} Christina Ernst, Jeremy Pike, Sarah J. Aitken, Hannah K. Long, Nils Eling, Lovorka Stojic, Michelle C. Ward, Frances Connor, Timothy F. Rayner, Margus Lukk, Robert J. Klose, Claudia Kutter, Duncan T Odom. 
Successful transmission and transcriptional deployment of a human chromosome via mouse male meiosis. \emph{eLife}, 5: e20235, 2016 
\end{Abstract}