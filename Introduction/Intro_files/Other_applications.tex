%!TEX root = ../intro.tex
%******************************
%	 Other applications of scRNAseq
%*****************************

\section{General applications of scRNA-Seq in biology}

The following section outlines the broad spectrum of research fields that benefit from the development of scRNA-Seq technologies. Whole transcriptomic read-outs of individual cells allowed the in-depth characterisation of embryonic development, hematopoiesis, immune responses, allowed the detection of rare cell types and lead to new insights into disease progression including cancer development. 

\subsection{Atlas-type approaches}

Until recently, scRNA-Seq technologies were used to generate transcriptomes of less than a thousand cells to address specific questions in cellular systems such as cell-type heterogeneity, allele-specific expression or pseudo-temporal trajectories in gene expression \citep{Kolodziejczyk2015review}. With the development of scRNA-Seq technologies that massively increased the throughput of cell capture and data generation, cellular composition of whole tissues and organisms can be assayed. The largest of these so called "atlases" to date is the 10X Genomics\textsuperscript{\textregistered}{} brain dataset comprising 1.3 million cells from embryonic mice. I was generated using 133 libraries sequenced on 11 Illumina HiSeq\textsuperscript{\textregistered}{} 4000 flowcells \citep{Note2017}. This experiment has been performed to exemplify the applicability of the commercial 10X genomics platform to generate more than 10 billion transcriptomes of individual cells across the human body as envisioned by the Human Cell Atlas Consortium \citep{Regev2017}.\\



\subsection{Developmental biology}

Embryonic development

Lineage tracing

Spatial transcriptomics

\todo{Gastrulation and early mouse development and what Tang is doing}
scGESTALT, MEMOIR, van Oudenaarden lineage tracing

\subsection{Evolutionary biology} 

\todo{describe recent papers from Heather}

\subsection{Immunology}

\todo{Sarah's work and Weizmann}

\subsection{Tissue function}

\todo{Mammary gland, liver}

\subsection{Cancer}

\todo{Avivs papers}