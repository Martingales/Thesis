%!TEX root = ../intro.tex
%******************************
%	 Sources of expression noise
%*****************************

\section{Sources of expression noise} 

Variability in expression across homogeneous populations of cells arises from intrinsic and extrinsic sources of noise (see Box 1). While intrinsic noise is gene-specific and therefore induces uncoordinated variation in RNA or protein expression between individual genes, extrinsic noise globally influences gene expression in each individual cell and therefore leads to co-variation across larger set of genes. Here, we give an overview on the different sources of intrinsic and extrinsic noise and discuss their beneficial or disruptive features in biological systems.

\subsection{Intrinsic noise}

Intrinsic noise in cell populations arises from stochasticity in biochemical reactions that are influenced by regulatory features on the DNA, epigenetic, chromatin, transcriptional, translational and post-translational level.   

\subsubsection{DNA}

Mutations in the DNA sequence can alter the binding affinity of transcription factors and therefore the rate at which a gene is expressed. A systematic study of the TDH3 gene expression in yeast found that mutations in known transcription factor binding sites (TFBS) decrease mean expression and increase expression noise. Moreover, Metzger et al. showed that evolutionary selection removes mutations that increase expression noise and that SNPs with large effects on expression noise show the lowest frequency within samples yeast strains(Metzger et al., 2015).
One of the most widely studied DNA motifs in relation to transcriptional noise is the TATA-box motif in promoters. Generally, TATA-box containing promoters show high levels of transcriptional noise(Faure, Schmiedel and Lehner, 2017) possibly due to a simple activation cycle containing one or few inactive states(Zoller et al., 2015). Stress response genes are enriched for the TATA-box motif and allow an early adjustment to changing environmental conditions(López-Maury, Marguerat and Bähler, 2009). Moreover, TATA-box containing genes show and increased interspecies variability(Tirosh et al., 2006) and higher spontaneous mutational variation(Landry et al., 2007) indicating an increased evolvability of these particular genes. In an early study, Raser et al. studied the noisy expression controlled by the budding yeast PHO5 promoter. This promoter contains the TATA-box motif and they showed that transcriptional noise is reduced when a mutational modification decreases the TATA-box strength(Raser and O’Shea, 2004). A more resent study confirmed this result and found mutations in yeast promoters that eliminate the TATA-box motif and therefore reduce noise levels in these genes(Hornung et al., 2012).
A possible confounding factor for the increased noise of TATA-box containing promoters is the number of TFBS. Tirosh et al. detected a two-fold enrichment of TBFS in TATA-box containing promoters(Tirosh et al., 2006). A later study showed that transcriptional noise scales with increased numbers of TFBS(Sharon et al., 2014). Furthermore, TATA-box containing genes lack enhancing histone marks and their increased variability in expression can therefore be explained by repressed chromatin(Choi and Kim, 2008).  
Promoters can be classified based on their shape as narrow with few transcriptional start sites (TSS) that predominantly control tissue-specific gene expression and broad promoters with larger numbers of TSS that control the expression of house keeping genes. Mutations that alter the shape of promoters increase transcriptional noise. Naturally occurring variants that increase noise are buffered by the heteroallelic haplotype they reside in(Schor et al., 2017). Furthermore, promoters with one or few TSS show higher levels of expression variability(Faure, Schmiedel and Lehner, 2017).
In addition to SNPs, copy number variations in parts of the genome influence gene expression [ref]. Combined analysis of DNA and RNA has shown that genes with low copy number tend to be noisier expressed compared to genes with multiple copies(Siddharth S. Dey et al., 2015). In the context of monoallelic expression, genes located on the X chromosome show increased mRNA half-life. Increased transcript stability leads to similar noise levels between X chromosomal genes and autosomal genes(Faure, Schmiedel and Lehner, 2017).

\subsubsection{Epigenetic factors}

Epigenetic factors as modifications of the DNA and histones as well as chromatin features (e.g. nucleosome positioning and long-range interactions) have been described to strongly affect transcriptional noise in biological systems.
CpG islands are genomic sites where DNA methylation preferentially occurs and different regulatory effects are observed between promoter and gene body methylations [ref, ref].  Recently, the presence of CpG islands in gene bodies but also at the TSS and in promoter regions was linked to a reduction in transcriptional noise(Faure, Schmiedel and Lehner, 2017). This is in line with previous findings that expression noise scales negatively with gene body methylation(Huh et al., 2013).
Modifications of histones induce the activation or repression of chromatin and  therefore indirectly modulates gene expression [Suganuma 2011, Annu. Rev. Biochem]. In an extensive study to link histone modifications to transcriptional noise, Faure et al. detected several histone modifications in promoter/core promoter motifs, at the TSS and in gene bodies to increase or decrease noise.  The repressive H3K27me3 mark is linked to higher noise levels when present at the TSS, in promoters/core promoters and in gene bodies. H3K4me1 only increases noise when present at the TSS and in the core promoter sequence while H3K9me3 increases noise when present in the promoter motif. H3K4me3, H3K9ac and H3K36me3 are linked to low levels of noise when present in gene bodies. In addition to these single features, bivalent promoters that carry the repressive H3K27me3 and enhancing H3K4me3 marks show high levels of transcriptional noise(Faure, Schmiedel and Lehner, 2017). 
Polycomb repressive complexes that silence transcription together with active RNA polymerase II (RNAPII) can bind these bivalent promoters. Switching between the repressed and active states then introduces gene expression variability across a population of cells(Kar et al., 2017). Additionally, deletions of different histone deacetylation complexes (Set3C and Rpd3(L)), that both repressed transcription by removing H3K9ac, showed different effects on transcriptional bursting(Weinberger et al., 2012).  Confirming this, a later study showed that gene body and promoter histone modifications independently influence burst size and burst frequency and therefore regulate mean expression and noise in an uncoupled fashion(Wu et al., 2017). These results indicate a fine-tuned regulation of transcriptional noise and support the functional role of biological noise in cell populations.
Chromatin structure dictates the expression of individual and clusters of genes and it’s role in transcriptional noise has widely been studied. Tiosh et al. showed that promoters with high nucleosome occupancy close to the TSS tend to display a high range of expression levels across varying conditions (transcriptional plasticity). Distant nucleosome-rich regions are on the other hand associated with low transcriptional noise(Tirosh and Barkai, 2008). Nucleosome covered promoters display shorter transcriptional rates, which in turn explains increased transcriptional noise for these promoters(S. S. Dey et al., 2015). Single-cell measures indicate cell-to-cell variations in nucleosome positioning around the PHO5 promoter upon stress induction. Even in the non-stressed state, a small fraction of cells exhibit nucleosome free regions at the promoter which explains low and possibly noisy expression of PHO5(Small et al., 2014). Deletion of chromatin remodeling complexes that remove nucleosomes upon transcription factor binding results in increased transcriptional noise compared to wild-type cells(Raser and O’Shea, 2004). 
Recent studies suggest that long-range enhancer-promoter interactions modulate transcriptional noise. Interference of CTCF-mediated enhancer-promoter contact either by CTCF knock-out or CTCF-binding site deletion leads to increased expression variability in selected genes(Ren et al., 2017).  In ESCs, genes within super-enhancer loci controlling pluripotency master regulators show high levels of noise. Down-stream targets of these master regulators show similar co-variation across ESCs(Faure, Schmiedel and Lehner, 2017).
Moreover, the positioning of genes on the genome controls expression noise with densely clustered genes being less noisy expressed in comparison to non-clustered genes(Kustatscher, Grabowski and Rappsilber, 2017). Additionally, genes positioned next to “noisy” genes display higher levels of transcriptional variability compared to genes that are located in proximity to “stable” genes(Kar et al., 2017). Expression noise is also increased for genes that are located in a repressed neighborhood, namely active genes in constitutive lamina-associated domains(Faure, Schmiedel and Lehner, 2017). 

\subsubsection{Transcription}

Transcription is initiated by transcription factors (TFs) binding to specific sites in the promoter region followed by recruitment of RNA polymerase II (RNAP2), RNA synthesis and RNA degradation. As discussed above, promoter architecture, namely the location and accessibility of TFBS and RNAP2 binding sites, dictates mean expression and transcriptional noise. 
In bacteria, the intracellular physical distance between TF source and the promoter sequence influences expression variability. TF expression proximal to their target genes results in less noisy expression compared to regulator sources distant to the promoter sequence(Goñi-Moreno et al., 2017). Once TFs bind to their target sequence, Carey et al. showed that the mean expression to noise ratio is promoter dependent while in the majority of cases, noise negatively scales with mean expression(Carey et al., 2013).   
Similar to TF binding dynamics, the assembly of RNAP2 complexes modulates transcriptional noise. An early study identified the connection between paused RNAP2 and synchronous expression of target genes. Genes without pre-loaded RNAP2 show more stochastic activation patterns(Boettiger and Levine, 2009). This finding has later been confirmed using scRNAseq data while increased variability was detected for genes with non-pause RNAP2 across the full range of expression levels(Day et al., 2016). 

\subsubsection{Post-transcriptional and translation}

On the post-transcriptional and translational level, mRNA localization, structure, degradation and translation have been shown to influence cell-to-cell variations in protein abundance. 
Upon transcriptional activation, mRNAs are produced in burst-like patterns while burst frequency modulates mean expression and noise and burst size influences solely mean expression(Hornung et al., 2012). While bursty transcript synthesis introduces stochastic fluctuations in nuclei between cells, active export of mRNAs into the cytoplasm can dampen this source of variability(Battich, Stoeger and Pelkmans, 2015). Reduces cytoplasmatic noise has also been shown for two nuclearly retained genes in the mammalian liver. Furthermore, this mode of noise control was proposed to be active across a range of metabolic tissues(Bahar Halpern, Caspi, et al., 2015).
Within the cytoplasm, mRNAs are subject to translation or degradation. At this stage, stochasticity from bursty gene expression is propagated to variation in protein abundance. The availability of mRNAs for translation is not only dictated by their syntheses but also their degradation rate. mRNA degradation is accelerated by recognition of micro RNAs. This process has been shown to preferentially reduce noise levels for lowly expressed genes in mouse ESCs to possibly retain cellular identity(Schmiedel et al., 2015). Conversely, temporal averaging of long-lived transcripts reduces noise in mRNA abundance. In that way, increased transcript stability compensates for noise introduced by the single-allele expression of genes on chromosome X(Faure, Schmiedel and Lehner, 2017).  
In addition to noise introduced by stochastic processes on the transcriptional level, the recognition and binding of ribosomes to mRNAs for translation initiation features a source for variations in protein abundance. Modulating translational efficiency by mutating the ribosome binding site and initiation codon showed an interaction between translation and variation in protein abundance(Ozbudak et al., 2002). Additionally, mRNA secondary structure formed by stem loops and ploy(G) motifs affects translation initiation and increases noise in protein levels(Dacheux et al., 2017).

\subsection{Extrinsic noise}

Extrinsic noise within a cell population arises from cells being in different states where differences in cellular components introduce variation in mRNA and protein abundance. Examples for cell states in otherwise homogeneous populations are characterized by differences in metabolism, cell cycle, volume, cell-to-cell and environmental signaling as well as cell density. It has been shown that extrinsic noise forms a major contribution to variations in gene expression and that transcript distributions can be predicted from the cellular state, population context and microenvironment(Battich, Stoeger and Pelkmans, 2015).

\subsubsection{Cell cycle}

Cell cycle has been widely discussed to form a crucial source of extrinsic noise(Colman-Lerner et al., 2005; Newman et al., 2006). In yeast populations, differences in transcriptional activities between the G1 and S/G2/M phases of the cell cycle lead to large-scale transcriptional heterogeneity across cell populations(Zopf et al., 2013). Under nutrient-poor conditions, growth rate is reduced and noise is elevated due to cells being in different cell-cycle stages(Keren et al., 2015).  Even under optimal growth conditions for mouse ESCs (2i media), cell cycle related genes show strong heterogeneity in expression across the cell population(Aleksandra A. Kolodziejczyk et al., 2015). Even though cell cycle is a major contributor to extrinsic noise, gating of cells into similar cell cycle stages did not drastically reduce noise levels(Raser and O’Shea, 2004). Nevertheless, cell cycle induced extrinsic noise can mask more subtle transcriptional heterogeneity and computational correction for this confounding effect can enhance the underlying noise signal(Buettner et al., 2015). 

\subsubsection{Cell volume}

Cellular volume provides another explanation for global differences in mRNA content between individual cells introducing large-scale transcriptional heterogeneity. Even though cell volume changes during cell cycle progression, within each phase cell volume varied as much as across all phases. It has been shown that mRNA counts scale with cellular volume to maintain transcript concentrations within each cell(Kempe et al., 2015; Padovan-Merhar et al., 2015)[Zhurinsky 2010, Curr Bio]. To avoid this source of heterogeneity, normalization approaches correct for mRNA content between individual cells(Vallejos et al., 2017).

\subsubsection{Metabolism}

The effect of metabolic fluctuations has been studied in E. coli populations. Noise in the expression of metabolic enzymes propagates to biochemical reactions that they catalyze. Changes in metabolism are then coupled to varying growth rates of individual cells, which in turn introduces large-scale transcriptional heterogeneity in cell populations(Kiviet et al., 2014).  

\subsubsection{Expression capacity}

Fluctuations in the expression capacity of cells due to quantitative differences in RNAP2 or ribosomes induce global variability among the majority of proteins(Colman-Lerner et al., 2005).

\subsubsection{Cell signalling}

A different source of extrinsic noise is the intra- or inter-cellular signaling state of individual cells. Fluctuations in membrane bound or cytoplasmic proteins lead to inconsistent transmission of signaling stimuli as exemplified by variability in TRAIL-induced apoptosis(Spencer et al., 2009). Similarly, variations of regulators in the ERK signaling pathway introduce downstream variability in nuclear response. The degree of which nuclear ERK response is varied depends of the position of the regulator in the topology of the signaling pathway(Iwamoto, Shindo and Takahashi, 2016). In C. elegans, perturbation of the Wnt signaling pathway displayed different degrees of the key Hox gene for Q neuroblast migration, mab-5, expression variability. It has been proposed that extrinsic noise, in this case the strength of the Wnt signal, modulates intrinsic variation in the expression of mab-5(Ji et al., 2013). 

\subsubsection{Physical constrains}

Physical constrains on cell growth and the direct population context influence the state of individual cells(Battich, Stoeger and Pelkmans, 2015). Snijder et al. performed detailed imaging based analysis of adherent human cells that were infected with different viruses. Clathrin mediated endocytosis was most variable with low cell density leading to inefficient mouse hepatits virus infection. Dengue virus preferentially infects edge cells while simian virus 40 infection was decreased with large cell density(Snijder et al., 2009). These experiments indicate the importance of local cellular microenvironment and cell-cell contacts leading to heterogeneity in cell states. 

Also describe:
•	Neves das RP, Jones NS, Andreu L, Gupta R, EnverT, Iborra FJ. 2010. Con- necting variability in global transcription rate to mitochondrial vari- ability. PLoS Biol 8: e1000560.
•	Xu H, Sepúlveda LA, Figard L, Sokac AM, Golding I. 2015. Combining pro- tein and mRNA quantification to decipher transcriptional regulation. Nat Methods 12: 739–742.
