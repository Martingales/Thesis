%!TEX root = ../chapter3.tex
%******************************
%	 Introduction 
%*****************************

\section{Introduction}

Gametogenesis is the process during which haploid gametes form carrying one copy of the individuals DNA. Sexual reproduction requires the fusion of two gametes from the opposite sex to drive evolution and adaptation \citep{McDonald2016}. Spermatogenesis, the male version of gametogenesis, is a tightly regulated developmental process to generate mature sperm. 
During spermatogenesis, spermatogonial stem cells undergo a unidirectional differentiation programme to form mature spermatozoa. This process occurs in the epithelium of seminiferous tubules in the testis and is tightly coordinated to ensure the continuous production of mature sperm cells. In the mouse, the first step involves spermatogonial differentiation to form pre-leptotene spermatocytes \citep{Oakberg1971, DeRooij1973, DeRooij2000}. Pre-leptotene spermatocytes then commit to meiosis, a cell division programme that consists of two consecutive cell divisions to produce haploid cells. To accommodate homologous recombination between sister chromatids and chromosome synapsis \citep{Marston2004}, prophase of meiosis I is extremely prolonged, lasting several days in males. It and can be divided into four substages: leptonema (L), zygonema (Z), pachynema (P) and diplonema (D). Following the two consecutive cell divisions, haploid cells known as round spermatids (RS), then undergo a complex differentiation programme called spermiogenesis to form mature spermatozoa \citep{Oakberg1956}. \\

Spermatogenesis takes place in a highly orchestrated fashion, with tubules periodically cycling through twelve epithelial stages defined by the combination of germ cells present \citep{Oakberg1956}. The completion of one cycle takes 8.6 days in the mouse, and the overall differentiation process from spermatogonia to mature spermatozoa requires approximately 35 days \citep{Oakberg1956a}. Thus, four to five generations of germ cells are present within a tubule at any given time. In adult animals, each tubule resides in a different cycle stage meaning that at any given time-point, the continuum of germ cell types in present in the testis. The continuity of this differentiation process and the gradual transitions between spermatogenic cell types have made the isolation and thus the molecular characterisation of individual sub-stages during spermatogenesis difficult.\\

To fully elucidate the molecular genetics of germ cell development, it is crucial to sample the full spectrum of germ cells present in testes of adult animals. For this purpose, we employed an unbiased droplet-based single-cell RNA-Sequencing (scRNA-Seq) approach. The 10X Genomics\textsuperscript{\small{TM}} platform was used to generate hundreds of thousands of Gel beads in EMulsions (GEM). Around 80\% of generated oil droplets capture barcoded gel beads in 8 channels in parallel. Each barcode consists of a sequencing adapter and primer, a 14bp barcode from a pool of 750,000 barcodes, a 10bp UMI and a 30bp poly-dT oligotide to capture poly-A mRNA \citep{Zheng2017}. GEMs are fused with individual cells at a low concentration and cell lysis begins instantaneous. mRNA molecules are capture by the poly-dT barcode and enzymes needed for reverse transcription are released from the gel beads. Each cDNA therefore contains a transcript specific UMI and a GEM specific barcode making demultiplexing possible. Barcoded cDNA is pooled for PCR amplification and library preparation \citep{Zheng2017}. \\

We used the transcriptomic profiles of thousands of single germ cells to characterize the complex transcriptional dynamics of spermatogenesis at high-resolution. To confidently identify and label cell populations throughout the developmental trajectory, we profiled cells from juvenile testes during the first wave of spermatogenesis. In juveniles, spermatogenesis has only progressed to a defined developmental stage, and therefore allowed us to unambiguously identify the most mature cell type by comparison with adult. The correct labelling of cell types was then used to dissect differentiation processes such as meiosis and spermiogenesis. Furthermore, juvenile samples enriched for spermatogonia which allowed us to characterize spermatogonial differentiation. Another major developmental process during spermatogenesis is the inactivation and reactivation of the X chromosome, which is subject to transcriptional silencing as a consequence of asynapsis \citep{Turner2007}. By combining bulk and single-cell RNA-Seq approaches with chromatin profiling, we identified that \textit{de novo} activated X-linked genes carry distinct chromatin signatures with high levels of repressive H3K9me3 in spermatocytes. \\

Finally, after fully characterising the transcriptional changes during spermatogenesis, I used the regression model presented in the previous chapter to study changes in transcriptional variability over the differentiation time-course. To this end, I generated \emph{post hoc} posterior distribution of linear regression coefficients to statistically test whether individual genes increase or decrease in variability. Furthermore, the clustering of variability profiles showed that rapid transcriptional changes during differentiation can cause peaks in such variability profiles.\\

Overall, this chapter presents an in-depth characterization of mouse spermatogenesis, provides new insights into the epigenetic regulation of X chromosome reactivation in post-meiotic spermatids and offers an analysis approach to find dynamic changes of transcriptional variability over a differentiation time-course.
