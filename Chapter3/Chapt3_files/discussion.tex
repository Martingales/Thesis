%!TEX root = ../chapter3.tex
%******************************
%	 Discussion 
%*****************************

\section{Discussion}

The testes are among the most proliferative tissues in the adult body and ensure fertility via the continuous production of millions of sperm per day. In contrast to most developmental differentiation processes which require the profiling of cellular populations at several time points \citep{Kernfeld2018, Scialdone2016, Wagner2018}, spermatogenesis occurs in continuous waves throughout the reproductive life span, with all intermediate cell types that arise across the ~35 day differentiation program present in adult testes. This provided a powerful opportunity to capture and profile an entire differentiation process by profiling the transcriptomes of thousands of single-cells at a single time point. \\

We identified key developmental transitions within the differentiation trajectory by profiling the first wave of spermatogenesis. Because germ cells have only progressed to a defined developmental point, the differentiation trajectory was truncated, facilitating identification of the most mature cell type. Profiling spermatogenesis in juvenile animals also naturally enriched for rare cell-types that are under-represented in adults. Among these, spermatogonia are of particular interest as these cells not only sustain male fertility, but are also the origin of the vast majority of testicular neoplasms \citep{Bosl1997}. We obtained more than 1100 transcriptional profiles for spermatogonia, allowing the identification of specific cell clusters within this heterogeneous cell population thus greatly improving the resolution over previous studies that only studied adult testes \citep{Lukassen2018}. Furthermore, our approach also enriched for and facilitated characterisation of the complexity within testicular somatic cell types, thus providing a valuable resource for understanding tissue homeostasis.\\

Droplet-based scRNA-Seq can profile large number of cells simultaneously \citep{Klein2015, Macosko2015, Zheng2017}, but often captures cells with a wide range of transcriptional complexity. Consequently, droplet-based assays present a major computational challenge in distinguishing between (i) droplets contain transcriptionally inactive cells versus (ii) empty droplets that contain (background) ambient RNA. By using a stringent default threshold, we identified the majority of somatic and germ cell types in testes, similar to recent single-cell expression studies in mouse and human \citep{Lukassen2018, Xia2018}. In addition, we applied a new statistical method to identify cells from droplet-based data by comparing the ambient RNA profiles \citep{Lun2018}, and were able to identify transcriptionally inactive leptotene/zygotene spermatocytes. This allowed us to bridge the developmental transition between spermatogonia and spermatocytes, thus providing a more complete view of the continuum of germ cell differentiation.\\

Perturbations of the gene expression programme during spermatogenesis frequently result in male sterility by causing a maturation arrest \citep{Cooke2002}. It is therefore of great interest to identify genes with novel spermatogenesis-related functions. To this end, we further characterized the dynamic gene expression patterns underlying meiosis and spermiogenesis. Although transcription broadly increases during meiotic prophase, we detected a diverse set of regulatory behaviours, particularly among spermatocyte-specific genes. In the late stages of spermiogenesis, transcription ceases due to the histone-to-protamine transition. Remarkably, the transcripts from a large set of genes remain highly abundant even after transcriptional shut-down in elongating spermatids. Many of these genes are known to be essential for reproductive success and therefore our analysis likely reveals novel genes with roles in sperm maturation.\\

The transcriptional silencing of the sex chromosomes during meiosis and their subsequent partial re-activation post-meiosis is essential for male fertility \citep{Mahadevaiah2008}. Failure of meiotic sex chromosome inactivation (MSCI) results in the expression of spermatocyte-lethal genes, as demonstrated for two Y chromosome encoded genes zinc finger protein Y-linked (\textit{Zfy}) 1 and 2 \citep{Royo2010}. Our discovery that H3K9me3 is enriched during meiosis at spermatid-specific genes suggests a stronger, targeted repression in spermatocytes for a key subset of X-linked genes. The deposition of H3K9me3 is specific to MSCI in males, and is not observed during general meiotic silencing of unpaired chromosomes (MSUC) \citep{Cloutier2016, Taketo2013, Turner2004a}. Interestingly, when comparing the levels of meiotic silencing for the X chromosome in spermatocytes with the unpaired X chromosome in XO oocytes, the transcriptional silencing was stronger in males compared to females \citep{Cloutier2016}. The deposition of H3K9me3 is linked to more robust silencing of X-chromosomal genes; our finding that spermatid-specific genes are particularly enriched for H3K9me3 in spermatocytes suggests that their repression may be necessary for male fertility. \\

Such a requirement could arise from the opposing evolutionary forces acting on the X chromosome \citep{Rice1992}. Due to its hemizygosity in males, the X chromosome has been predicted to be enriched for male-specific genes. In contrast, meiotic silencing allows pachytene-lethal genes to survive on the X chromosome, since their deleterious effect will be masked by MSCI, similarly to \textit{Zfy1/2} on the Y chromosome \citep{Royo2010}. Our study thus raises interesting questions about how H3K9me3 is targeted to specific genes on the X chromosome in spermatocytes, and how transcription is reactivated in post-meiotic spermatids.


