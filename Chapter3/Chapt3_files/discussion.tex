%!TEX root = ../chapter3.tex
%******************************
%	 Discussion 
%*****************************

\section{Discussion}

The testes are among the most proliferative tissues in the adult body and ensure fertility via the continuous production of millions of sperm per day. Most developmental differentiation processes require the profiling of cellular populations at several time points \citep{Kernfeld2018, Scialdone2016, Wagner2018}. One of the exemptions is blood formation where commitment to different lineages can be profiled at once \citep{Dahlin2018}. Similarly, spermatogenesis occurs in continuous waves throughout the reproductive life span of animals. At any given time-point, all intermediate cell-types that arise across the ~35 day differentiation program are present in adult testes. This provided a powerful opportunity to capture and profile an entire differentiation process by profiling the transcriptomes of thousands of single-cells at a single time point. \\

We exploited the natural synchronisation of the first wave of spermatogenesis to identify key developmental transitions within the differentiation trajectory. Because germ cells have only progressed to a defined developmental point, the differentiation trajectory was truncated, facilitating identification of the most mature cell type.
In contrast, Chen \emph{et al.}, 2018 sorted synchronised spermatocyte and spermatid populations after blocking spermatogenesis with WIN 18,446. This allowed a strict enrichment for cells in specific stages during spermatogenesis but lost the natural trajectory of this continuous differentiation process \citep{Chen2018}. Profiling spermatogenesis in juvenile animals also naturally enriched for rare cell-types that are under-represented in adults. In the case of haematopoiesis, cells need to be sorted to capture otherwise under-represented cell types \citep{Dahlin2018}. Among these rare cell types, spermatogonia are of particular interest as these cells not only sustain male fertility, but are also the origin of the vast majority of testicular neoplasms \citep{Bosl1997}. We obtained more than 1,100 transcriptional profiles for spermatogonia, allowing the identification of specific cell clusters within this heterogeneous cell population thus greatly improving the resolution over previous studies that only studied adult testes \citep{Lukassen2018}. Furthermore, our approach also enriched for and facilitated characterisation of the complexity within testicular somatic cell types. Among those are characteristic immune cells and precursor cells that only exist until a few days after birth.\\

Droplet-based scRNA-Seq can profile large number of cells simultaneously \citep{Klein2015, Macosko2015, Zheng2017}, but often captures cells with a wide range of transcriptional complexity. Consequently, droplet-based assays present a major computational challenge in distinguishing between (i) droplets contain transcriptionally inactive cells versus (ii) empty droplets that contain (background) ambient RNA. By using a stringent default threshold, we identified the majority of somatic and germ cell types in testes, similar to recent single-cell expression studies in mouse and human \citep{Lukassen2018, Xia2018, Chen2018}. In addition, we applied a statistical method to identify cells from droplet-based data by comparing the ambient RNA profiles \citep{Lun2018}, and were able to identify transcriptionally inactive leptotene/zygotene spermatocytes. This allowed us to bridge the developmental transition between spermatogonia and spermatocytes, thus providing a more complete view of the continuum of germ cell differentiation.\\

After the in-depth characterisation of germ and somatic cell-types in adult testes, we profiled major developmental processes during mouse spermatogenesis. During meiosis, we detect the expression of hundreds of genes as being associated with the developmental trajectory. Some of these genes show a sterility phenotype when perturbed and we reason that this is also the case for the majority of genes that follow the developmental trajectory in expression. Spermiogenesis is characterised by wide-scale chromatin rearrangements and we detect a clear increase in testis-specific histone variants, transition proteins and protamines during late stages of sperm maturation. Again, genes that follow this trend could be important regulators that cause sterility upon misexpression.   \\

The transcriptional silencing of the sex chromosomes during meiosis and their subsequent partial re-activation post-meiosis is essential for male fertility \citep{Mahadevaiah2008}. Failure of meiotic sex chromosome inactivation (MSCI) results in the expression of spermatocyte-lethal genes, as demonstrated for two Y chromosome encoded genes: zinc finger protein Y-linked (\textit{Zfy}) 1 and 2 \citep{Royo2010}. Our discovery that H3K9me3 is enriched during meiosis at spermatid-specific genes suggests a stronger, targeted repression in spermatocytes for a key subset of X-linked genes. The deposition of H3K9me3 is specific to MSCI in males, and is not observed during general meiotic silencing of unpaired chromosomes \citep{Cloutier2016, Taketo2013, Turner2004a}. Our finding that spermatid-specific genes are particularly enriched for H3K9me3 in spermatocytes suggests that their repression may be necessary for male fertility. \\

When profiling changes in variability over the differentiation trajectory, I detected a strong confounding effect between the variability measure and the correlation between expression and pseudo-time. Therefore, new measures of variability need to be derived to account for this dependency. For example, graph-based measures can assign a variability measure for each cell when comparing expression across a local neighbourhood. Next, fitting a generalized linear model between these variability estimates and the ordering of cells along pseudotime can be used to detect changes in variability. Nevertheless, confounding effects such as the expression level can obstruct such analysis.

